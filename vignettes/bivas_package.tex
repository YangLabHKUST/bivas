% -*- mode: noweb; noweb-default-code-mode: R-mode; -*-
\documentclass[11pt]{article}
%% Set my margins
\setlength{\oddsidemargin}{0.0truein}
\setlength{\evensidemargin}{0.0truein}
\setlength{\textwidth}{6.5truein}
\setlength{\topmargin}{0.0truein}
\setlength{\textheight}{9.0truein}
\setlength{\headsep}{0.0truein}
\setlength{\headheight}{0.0truein}
\setlength{\topskip}{0pt}
%% End of margins

% \usepackage{subfigure}

%%\pagestyle{myheadings}
%%\markboth{$Date$\hfil$Revision$}{\thepage}
\usepackage[pdftex,
bookmarks,
bookmarksopen,
pdfauthor={Mingxuan Cai},
pdftitle={bivas Vignette}]
{hyperref}

\title{Quick start for `\texttt{bivas}' Package}
\author{Mingxuan Cai$~^1$, Mingwei Dai$~^{2,4}$, Jingsi Ming$~^1$, \\
Heng Peng$~^1$, Jin Liu$~^3$, and Can Yang$~^4$\\
\\
$~^1$ Department of Mathematics, Hong Kong Baptist University, Hong Kong.\\
$~^2$ School of Mathematics and Statistics, Xi'an Jiaotong University, Xi'an, China.\\
$~^3$ Centre of Quantitative Medicine, Duke-NUS Graduate Medical School, Singapore.\\
$~^4$ Department of Mathematics, The Hong Kong University of Science and Technology,\\
Hong Kong.\\
}

\date{\today}



\usepackage{Sweave}
\begin{document}
\input{bivas_package-concordance}
\maketitle

\section{Overview}

This vignette provides an introduction to the `\texttt{bivas}' package.
R package `\texttt{bivas}' implements BIVAS model,
which is an efficient statistical approach for bi-level variable selection. It provides model parameter estimation as well as statistical inference.

The package can be loaded with the command:


\begin{Schunk}
\begin{Sinput}
R> library("bivas")
\end{Sinput}
\end{Schunk}

This vignette is organized as follows.
Section \ref{bivas} discusses how to fit BIVAS in linear regression with group structure.
Section \ref{bivas_mt} discusses how to fit BIVAS in multi-task learning.
Section \ref{inference} explains command lines for statistical inference for identification of associated variables and groups using BIVAS.


Please feel free to contact Can Yang at \texttt{macyang@ust.hk} for any questions or suggestions regarding the `\texttt{bivas}' package.

\section{Workflow}\label{workflow}

R package \texttt{bivas} provides fitting function for both linear regression with group structure and multi-task learning;
it is also designed with an option to execute prarallel computing. In this vignette, we use the simulated groupExample and mtExample in the package.

\subsection{Fitting the BIVAS in linear regression with group structure}\label{bivas}

In this section, we use groupExample data as demonstration. We set the number of groups, variables and sample size to be $K=100$, $p=1,000$ and $n=500$ respectively.

\begin{Schunk}
\begin{Sinput}
R> data(groupExample)
R> group <- groupExample$group
R> X <- groupExample$X
R> y <- groupExample$y
R> length(group)
\end{Sinput}
\begin{Soutput}
[1] 1000
\end{Soutput}
\begin{Sinput}
R> length(unique(group))
\end{Sinput}
\begin{Soutput}
[1] 100
\end{Soutput}
\begin{Sinput}
R> dim(X)
\end{Sinput}
\begin{Soutput}
[1]  500 1000
\end{Soutput}
\begin{Sinput}
R> length(y)
\end{Sinput}
\begin{Soutput}
[1] 500
\end{Soutput}
\end{Schunk}

The data to be used are `\texttt{X}', `\texttt{y}' and `\texttt{group}'. `\texttt{X}' and `\texttt{y}' are typical design matrix and response vector for linear regression; `\texttt{group}' is a vector containing the group membership of each variable. The length of `\texttt{group}' is assumed to be the same as the number of columns of matrix provided to `\texttt{X}', indicating which group each variable belongs to.

Now we can fit BIVAS with the function:
\begin{Schunk}
\begin{Sinput}
R> fit_bivas <- bivas(y = y, X = X,group = group)
\end{Sinput}
\end{Schunk}
if the computation should be parallelized with 2 threads, we can provide the number of threads in an additional argument as follows:
\begin{Schunk}
\begin{Sinput}
R> fit_bivas <- bivas(y = y, X = X, group = group, coreNum = 2)
\end{Sinput}
\end{Schunk}
with this, the computation tasks are dynamically allocated to 2 threads in the fitting process.

The returned object is `\texttt{fit\_bivas}' of S3 class containing parameter estimation, posterior inclusion probability, posterior mean and evidence lower bound for each EM procedure. Statistical inference can be made using the command in \ref{inference}.

To evaluate the model perfomance, use
\begin{Schunk}
\begin{Sinput}
R> perf_bivas <- perf.bivas(y = y, X = X, group = group, coreNum = 2, nfolds=10)
\end{Sinput}
\begin{Soutput}
start cv process......... total 10 validation sets 
1 -th validation set... 
2 -th validation set... 
3 -th validation set... 
4 -th validation set... 
5 -th validation set... 
6 -th validation set... 
7 -th validation set... 
8 -th validation set... 
9 -th validation set... 
10 -th validation set... 
\end{Soutput}
\begin{Sinput}
R> str(perf_bivas)
\end{Sinput}
\begin{Soutput}
List of 3
 $ cvm    : num 3.62
 $ cvsd   : num 0.515
 $ testErr: num [1:10] 4.13 3.78 3.47 3.29 3.28 ...
\end{Soutput}
\end{Schunk}
which returns a list of vectors containing the estimates of: cross-validation mean squared error, standard error of mse and the test error for each trial. By default we use 10 fold cross-validation (`\texttt{nfolds=10}').


\subsection{Fitting the BIVAS in multi-task learning}\label{bivas_mt}
For multi-task learning, we use mtExample as demonstration. We set the variables, number of tasks and sample sizes to be $K=500$, $L=3$ and $n_1=500$, $n_2=600$, $n_3=500$ respectively.

\begin{Schunk}
\begin{Sinput}
R> data(mtExample)
R> X <- mtExample$X
R> y <- mtExample$y
R> class(X)
\end{Sinput}
\begin{Soutput}
[1] "list"
\end{Soutput}
\begin{Sinput}
R> class(y)
\end{Sinput}
\begin{Soutput}
[1] "list"
\end{Soutput}
\begin{Sinput}
R> sapply(X,ncol)
\end{Sinput}
\begin{Soutput}
[1] 500 500 500
\end{Soutput}
\begin{Sinput}
R> length(X)
\end{Sinput}
\begin{Soutput}
[1] 3
\end{Soutput}
\begin{Sinput}
R> sapply(y,length)
\end{Sinput}
\begin{Soutput}
[1] 500 600 500
\end{Soutput}
\end{Schunk}
where the data to be used are are stored in the lists `\texttt{X}' and `\texttt{y}'. The length of `\texttt{X}' is assumed to be the same as the length of `\texttt{y}'.

To fit the model, the function `\texttt{bivas\_mt}' is used:
\begin{Schunk}
\begin{Sinput}
R> fit_bivas_mt <- bivas_mt(y = y, X = X)
\end{Sinput}
\end{Schunk}

Again, parallel computing can be called by the following:
\begin{Schunk}
\begin{Sinput}
R> fit_bivas_mt <- bivas_mt(y = y, X = X, coreNum = 2)
\end{Sinput}
\end{Schunk}

`\texttt{fit\_bivas\_mt}' is an S3 object containing parameter estimation, posterior inclusion probability, posterior mean and evidence lower bound for each EM procedure.


To evaluate the model perfomance, use
\begin{Schunk}
\begin{Sinput}
R> perf_bivas_mt <- perf.bivas_mt(y = y, X = X, coreNum = 2, nfolds=10)
\end{Sinput}
\begin{Soutput}
start cv process......... total 10 validation sets 
1 -th validation set... 
2 -th validation set... 
3 -th validation set... 
4 -th validation set... 
5 -th validation set... 
6 -th validation set... 
7 -th validation set... 
8 -th validation set... 
9 -th validation set... 
10 -th validation set... 
\end{Soutput}
\begin{Sinput}
R> str(perf_bivas)
\end{Sinput}
\begin{Soutput}
List of 3
 $ cvm    : num 3.62
 $ cvsd   : num 0.515
 $ testErr: num [1:10] 4.13 3.78 3.47 3.29 3.28 ...
\end{Soutput}
\end{Schunk}
which returns a list of vectors containing the estimates of: cross-validation mean squared error, standard error of mse and the test error for each trial. By default we use 10 fold cross-validation (`\texttt{nfolds=10}').


\subsection{Statistical inference, variable selection and prediction}\label{inference}

In this section we show how to make statistical inference based on the fitted `\texttt{bivas}' object.

To obtain the posterior inclusion probability of groups and variables, use
\begin{Schunk}
\begin{Sinput}
R> pos_bivas <- getPos(fit_bivas,weight=T)
R> str(pos_bivas)
\end{Sinput}
\begin{Soutput}
List of 2
 $ var_pos  : num [1:1000, 1] 0.00296 0.00296 0.00298 0.00296 0.00296 ...
 $ group_pos: num [1:100, 1] 0.0078 1 0.0131 0.00607 0.00898 ...
  ..- attr(*, "dimnames")=List of 2
  .. ..$ : chr [1:100] "1" "2" "3" "4" ...
  .. ..$ : NULL
\end{Soutput}
\begin{Sinput}
R> pos_bivas_mt <- getPos(fit_bivas_mt,weight=T)
R> str(pos_bivas_mt)
\end{Sinput}
\begin{Soutput}
List of 2
 $ var_pos  : num [1:500, 1:3] 0.0114 0.0265 0.0112 0.026 0.0171 ...
 $ group_pos: num [1:500, 1] 0.937 0.912 0.901 0.903 0.901 ...
\end{Soutput}
\end{Schunk}
which returns a list of values between 0 and 1 indicating the posterior inclusion probabilities of groups and variables. If `\texttt{weight=F}', the outcomes of all EM procedures will be returned unweighted by the lower bound. For multi-task BIVAS, the variable level results are stored in a K by L matrix.

To obtain the coefficient estimates, use
\begin{Schunk}
\begin{Sinput}
R> coef_bivas <- coef(fit_bivas,weight=T)
R> str(coef_bivas)
\end{Sinput}
\begin{Soutput}
List of 2
 $ cov : num [1, 1] 6.41e-17
  ..- attr(*, "dimnames")=List of 2
  .. ..$ : chr "Intercept"
  .. ..$ : NULL
 $ beta: num [1:1000, 1] -3.37e-05 1.92e-05 -5.93e-04 2.04e-04 -2.22e-04 ...
\end{Soutput}
\begin{Sinput}
R> coef_bivas_mt <- coef(fit_bivas_mt,weight=T)
R> str(coef_bivas_mt)
\end{Sinput}
\begin{Soutput}
List of 2
 $ cov : num [1, 1:3] 5.05 5.02 5
  ..- attr(*, "dimnames")=List of 2
  .. ..$ : chr "Intercept"
  .. ..$ : NULL
 $ beta: num [1:500, 1:3] 0.000394 0.001997 -0.000347 -0.001942 0.001002 ...
\end{Soutput}
\end{Schunk}
which returns a list of vectors containing the estimates of fixed effects and random effects. If `\texttt{weight=F}', the outcomes of all EM procedures will be returned unweighted by the lower bound. For multi-task BIVAS, the variable level results are stored in a K by L matrix.

To obtain the fdr, apply generic function `\texttt{fdr}' to a fitted `\texttt{bivas}' object:
\begin{Schunk}
\begin{Sinput}
R> fdr_bivas <- fdr(fit_bivas,FDRset=0.05,control="global")
R> str(fdr_bivas)
\end{Sinput}
\begin{Soutput}
List of 2
 $ FDR     : num [1:1000] 0 0 0 0 0 0 0 0 0 0 ...
 $ groupFDR: Named num [1:100] 0 1 0 0 0 0 0 0 0 0 ...
  ..- attr(*, "names")= chr [1:100] "1" "2" "3" "4" ...
\end{Soutput}
\begin{Sinput}
R> fdr_bivas_mt <- fdr(fit_bivas_mt,FDRset=0.05,control="global")
R> str(fdr_bivas_mt)
\end{Sinput}
\begin{Soutput}
List of 2
 $ FDR     : num [1:500, 1:3] 0 0 0 0 0 0 0 0 0 0 ...
 $ groupFDR: num [1:500] 1 1 0 0 0 0 0 0 0 1 ...
\end{Soutput}
\end{Schunk}
which returns a list of binary values indicating the relevant groups and variables. One can specify threshold in FDR control through the argument `\texttt{FDRset}' which is 0.05 in our case; and we allow both local (`\texttt{control="local"}') and global (`\texttt{control="global"}') FDR controls. For multi-task BIVAS, the variable level results are stored in a K by L matrix.

To predict with new data based on fitted model, use
\begin{Schunk}
\begin{Sinput}
R> y_hat <- predict(fit_bivas, X=groupExample$X)
R> str(y_hat)
\end{Sinput}
\begin{Soutput}
 num [1:500, 1] 0.851 -1.257 -1.159 0.867 -2.581 ...
\end{Soutput}
\begin{Sinput}
R> y_hat <- predict(fit_bivas_mt, X=mtExample$X)
R> str(y_hat)
\end{Sinput}
\begin{Soutput}
List of 3
 $ : num [1:500, 1] 3.05 5.18 4.61 5.62 4.64 ...
 $ : num [1:600, 1] 6.13 5.42 3.36 4.9 5.42 ...
 $ : num [1:500, 1] 4.78 6.58 5.94 5.47 5.58 ...
\end{Soutput}
\end{Schunk}
which returns a vector of predicted values for group BIVAS or a list of predicted values for each task for multi-task BIVAS.




\begin{thebibliography}{99}
Mingxuan Cai, Mingwei Dai, Jingsi Ming, Heng Peng, Jin Liu and Can Yang. BIVAS: A scalable Bayesian method for bi-level variable selection with applications. 2017. Under review.


\end{thebibliography}

\end{document}
